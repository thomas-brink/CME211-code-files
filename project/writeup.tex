\documentclass[12pt, a4paper]{article}

\usepackage[ruled, vlined, linesnumbered, longend]{algorithm2e}
\usepackage{amsmath, amssymb}
\usepackage[margin=2cm]{geometry}

\linespread{1.25}

\title{\textbf{Writeup CME 211 Project Part 1}}
\author{Thomas Brink (06446502)}
\date{November 16, 2021}

\begin{document}

\maketitle

\section{Algorithm}
In Algorithm \ref{alg:CGSolver}, we provide pseudo-code for
the conjugate gradient algorithm we use to solve the linear
system. Note that, in this algorithm, we use \textbf{bold}
notation for vectors and matrices.

\begin{algorithm}[h]
    \setcounter{AlgoLine}{0}
    \KwData{Sparse CSR matrix $\mathbf{A}_{n\times n}$; 
    linear system  outcome $\mathbf{b}$; tolerance level 
    \textit{tol}}
    
    \KwResult{Number of iterations $i^*$ required to reach
    convergence; solution vector $\mathbf{u}^*$}
    
    Initialize $\mathbf{u}_0 = \mathbf{1}_n$; $i = 0$;
    $i_{max} = n$; $\mathbf{r}_0$ = $\mathbf{b} - 
    \mathbf{A}\mathbf{u}_0$; $\mathbf{p}_0 = \mathbf{r}_0$
    
    \While{$i < i_{max}$}{
        $i$ := $i+1$
        
        $\alpha := \dfrac{\mathbf{r}_{i-1}^\text{T}\mathbf{r}_{i-1}}
        {\mathbf{p}_{i-1}^\text{T}\mathbf{A}\mathbf{p}_{i-1}}$
        
        $\mathbf{u}_i = \mathbf{u}_{i-1} + \alpha\mathbf{p}_{i-1}$
        
        $\mathbf{r}_i = \mathbf{r}_{i-1} - \alpha\mathbf{A}\mathbf{p}_{i-1}$
        
        \If{$\dfrac{||\mathbf{r}_i||_2}{||\mathbf{r}_0||_2} <$ \textit{tol}}{
            $i^* := i$
            
            $\mathbf{u}^* = \mathbf{u}_i$
            
            break
        }
        
        $\beta := \dfrac{\mathbf{r}_i^\text{T}\mathbf{r}_i}
        {\mathbf{r}_{i-1}^\text{T}\mathbf{r}_{i-1}}$
        
        $\mathbf{p}_i = \mathbf{r}_i + \beta\mathbf{p}_{i-1}$
    }
    
    \caption{Conjugate gradient pseudo-code}
    \label{alg:CGSolver}
    \end{algorithm}

\newpage
\section{Code Breakdown}
In this project, we provide code for solving a linear sparse system
using conjugate gradient. The code is broken down in several parts,
which we will briefly discuss sequentially below. The majority of code
for implementing the conjugate gradient solver is captured by the
\textit{CGSolver.cpp} and \textit{matvecops.cpp} files, in addition
to which several other files implement further functionality.

\begin{itemize}
    \item \textbf{CGSolver.cpp}. In the \textit{CGSolver.cpp} file, we perform 
    the  conjugate gradient algorithm provided in Algorithm \ref{alg:CGSolver}. 
    This algorithm finds a solution $x^*$ to the system $Ax = b$, where $A$ is 
    sparse (CSR) and $b$ is given beforehand. Our code returns the number 
    of iterations until convergence. For this algorithm, we need to define
    several matrix and vector operations. Instead of doing so within the
    algorithm for each separate line, we make use of a separate file
    with functions for specific operations, called \textit{matvecops.cpp}.
    This avoids redundancy in the code as we avoid excess code for similar
    operations (e.g., lines 5 and 12 in Algorithm \ref{alg:CGSolver}).
    \item \textbf{matvecops.cpp}. As previously stated, in the 
    \textit{matvecops.cpp} file, we define useful
    matrix-vector operations for the conjugate gradient algorithm. 
    We break these operations down into several small functions that all 
    represent a single operation, which should make the code modular,
    less redundant, and easy to understand. The functions include:
        \begin{enumerate}
            \item \textit{vecSum}: this function performs vector addition 
            given two input vectors.
            \item \textit{vecScalarProd}: this function multiplies an input
            vector by a scalar.
            \item \textit{dotProduct}: this function computes the dot product
            between two input vectors.
            \item \textit{L2Norm}: this function computes the $l_2$-norm of
            a given vector.
            \item \textit{matVecProd}: this function computes the matrix-vector
            product of a CSR-type sparse matrix and a given vector.
        \end{enumerate}
    \item \textbf{COO2CSR.cpp}. In this function, which was provided by the 
    instructor, a COO matrix is transformed into a CSR matrix. We do so as
    the previous matrix-vector operations work well with the CSR format.
    \item \textbf{main.cpp}. In the \textit{main.cpp} file, we provide the 
    main functionality for running our code. That is, the main file i) takes 
    a matrix input file and a solution file name, ii) performs the conjugate 
    gradient algorithm (and the required prior computations, such as the COO 
    to CSR transform), iii) writes the solution $\mathbf{x}^*$ to 
    the solution file, and iv) prints a 'success' or 'error' message.
    \item \textbf{Header files}. Besides providing the code definitions in 
    the .cpp files, we provide function declarations in separate header files. 
    \textit{matvecops.hpp}, \textit{COO2CSR.hpp}, and \textit{CGSolver.hpp}.
    \item \textbf{makefile}. In addition to the above files, we provide a 
    makefile to compile our program.
\end{itemize}

\end{document}

\documentclass[12pt, a4paper]{article}
\usepackage[utf8]{inputenc}
\usepackage{fullpage}
\usepackage{graphicx}
\usepackage{amsmath, amssymb}
\usepackage[margin=2cm]{geometry}
\linespread{1}
\usepackage{caption}
\usepackage{algorithm}
\usepackage{algorithmic}
\usepackage{float}
\usepackage{wrapfig}
\setlength{\parskip}{0.7em}
\renewcommand{\labelenumi}{\alph{enumi})}
\makeatletter
\def\BState{\State\hskip-\ALG@thistlm}
\makeatother
\usepackage{subcaption}
\usepackage{multirow,booktabs,setspace,caption}

%Hyper references in PDF document
\usepackage{xcolor}
\usepackage{makeidx}
\definecolor{darkblue}{rgb}{0,0,0.55}
\usepackage{nameref}
\usepackage{hyperref}
\setcounter{secnumdepth}{1}
\hypersetup{colorlinks=true,linktocpage=true,linkcolor=darkblue,
            citecolor={darkblue},allcolors={darkblue},bookmarks=true}

\title{\textbf{README CME 211 HW4} \\ Thomas Brink (06446502)}
\date{\today}

\begin{document}

\maketitle

\section{Introduction}
In this homework, we provide code for a 'truss' class. This class takes
in input files on the beams and joints of a truss and analyzes the 
geometry and forces that are at play in the truss. We provide two
files: \textit{main.py} and \textit{truss.py}.

\section{main.py}
In the \textit{main.py} file, we provide the handling of user inputs and
call the truss class to provide output. That is, when running our code,
we call \textit{main.py} and some input arguments, which results in
some sort of output.

Here's an example of command line code to analyze the truss geometry
and forces given some user inputs:

\textit{\$ python3 main.py [joints file] [beams file] [optional plot file]}

Note that the plot file is optional. When given by the user, a figure
will be created that plots the truss geometry (2D). 

The output of the command line example is either a table containing
numbered Beams and the corresponding forces as columns or some
sensible error message. We provide more insight into these error
messages when discussing the \textit{truss.py} file. 

\section{truss.py}
In \textit{truss.py}, we provide code for the \textit{Truss} class.
This class allows users to create instances of a truss with given
joints and beams data. The code for this class consists of 6 functions
and some introductory import statements, which we consequently discuss
one by one.

\begin{enumerate}
    \item Import statements: we import the \textit{matplotlib}, \textit{numpy},
    \textit{scipy}, and \textit{warnings} modules. We use these model for 
    plotting, data handling and manipulation, (sparse) matrix algebra, and
    warning handling, respectively.
    
    \item \textit{\_\_init\_\_}: the first function is the 
    \textit{\_\_init\_\_} function, which we use to initialize 
    an instance of the truss class given the user inputs and to 
    call the other functions in the truss class so as to perform 
    our analysis on the initialized instance. The 
    \textit{\_\_init\_\_} function takes three inputs (not including
    \textit{self}); a file name for joints data, a file name for beams
    data, and an optional name to save the truss geometry plot to. By
    default, this argument is set to 0, which leads to not plotting the
    geometry.
    
    \item \textit{load\_data}: this second function we use loads in the
    joints and beams data given by the user by storing beams data in a
    numpy array and the joints data in a dictionary (with joint number
    as key and the data for this joint as provided in the input file
    as value). Also, we raise exceptions if errors occur while loading
    in the data and add the index of a joint to the value in the 
    dictionary (so as to make ordering of the joints easier, which
    we will use in constructing the matrix system for our analysis).
    
    \item \textit{plot\_geometry}: this function plots the x- and
    y-coordinates of joints and plots lines between these points
    in case a beam between these joints exists. We save the plot
    to a file with the name that is given by the user as input. 
    
    \item \textit{calculate\_forces}: this function is the main
    calculation function in the class, and is used to construct
    the vector of external forces and matrix of force coefficients
    that are used in the linear system 
    \begin{equation}
    \label{eq: linear-system}
        Ax = b,
    \end{equation}
    where $A$ is the matrix with force coefficients, $b$ the vector
    of external forces, and $x$ a vector with beam and reaction forces.
    Note that $x$ is unknown and finding $x$ is the main goal of the
    class. The dimensions of $A$ should be $n \times n$, where
    $n = 2 \times$ (the number of joints), or $n =$ (the number of
    beams) $+ 2 \times$ (the number of joints with a reaction force.
    
    In the \textit{calculate\_forces} function, we check whether these
    dimensions are correct and raise an exception otherwise. Namely, if
    the dimensions don't add up, we are not able to find $x$. Note that
    $x$ is a vector with beam forces and reaction forces, while $b$ is
    a vector with external forces for each x- and y-component of each
    joint. 
    
    We compute $b$ by creating a sparse matrix that reads in the external
    force data in the joints data file. We compute $A$ coefficients for
    beam forces by calculating the difference in x- and y-components
    between joints connecting a beam, scaled by the length of the beam.
    For the starting joint, we compute
    \begin{equation}
        \text{coeff} = \frac{x_1 - x_2}{\text{length of beam}},
    \end{equation}
    and similar for the y-component. For the ending joint of a beam, we
    compute
    \begin{equation}
        \text{coeff} = \frac{x_2 - x_1}{\text{length of beam}}.
    \end{equation}
    
    After filling in the beam force coefficients, we only need the
    coefficients for the reaction forces. In our joints input data,
    we have the \textit{zerodisp} column that indicates which joints
    have a reaction force. We add a coefficient $1$ to the x- and
    y-component of a joint in the respective reaction force column
    of $A$. Just as with $b$, we save $A$ in sparse format. We use
    a $csr$-type sparse matrix, which allows us to immediately go
    to the next function, which solves the linear system.
    
    \item \textit{compute\_static\_equilibrium}: this function
    uses the matrix of force coefficients $A$ and the vector
    of external force coefficients $b$ to solve for $x$ in linear
    system \ref{eq: linear-system}. In case the coefficient 
    matrix is singular ($\det = 0$), we cannot solve the system
    and raise an exception. Otherwise, we solve the system and
    save the forces in vector $x$.
    
    \item \textit{\_\_repr\_\_}: the last function is the string
    representation function, which allows us to print the output
    of our truss analysis directly when using the $print(instance)$
    command. We use the specific formatting as outlined in the
    assignment.
\end{enumerate}

Putting the two code files together, we have obtained a complete
set of code that allows us to perform analysis on a truss instance
given joints and beams data!

\end{document}
